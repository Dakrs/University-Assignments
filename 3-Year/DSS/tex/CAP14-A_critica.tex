\section{Análise Crítica}
Na primeira fase do trabalho pediram-nos para fazer uma análise mais abstrata de modo a criar uma primeira imagem e ideia sobre a aplicação que estamos a desenvolver. A nossa abordagem começou pelo desenvolvimento do modelo domínio. Aqui tentamos estabelecer as principais entidades e as relações entre as mesmas de modo a criar a nossa primeira interpretação sobre o problema. Inicialmente, deparámo-nos com alguns erros de interpretação do enunciado como a função do cliente na aplicação que inicialmente achávamos que ia ser o principal utilizador e mais tarde foi alterado. Esta parte da modulação fez como que a nossa visão sobre o que queríamos para a aplicação ficasse mais clara e expressa de uma forma mais simplista. Ao modelo de domínio seguiu-se o desenvolvimento dos Use Case. Quando começamos o desenvolvimento dos mesmos, focámo-nos essencialmente em descrever a ação entre os atores e a aplicação. Esta descrição foi demasiado detalhada, fornecia informação que não era precisa e sobretudo limitava muito aquilo que queríamos fazer da aplicação. Como ainda nos encontrávamos numa fase inicial decidimos refletir melhor sobre a nossa solução e chegamos à conclusão que eram limitações que não queríamos ter. A isto seguiu-se uma reestruturação do problema de uma forma mais genérica que descrevia, na mesma, todas as funcionalidades dando-nos uma margem maior para aquilo que pode vir a ser a nossa aplicação.
Seguidamente demos inicio ao desenvolvimento das máquinas de estado. Aqui tentamos recriar os diversos estados da nossa aplicação e entender como é que estes alternavam entre si. O nosso principal obstáculo foi a dificuldade em exprimir os estados durante a personalização de uma configuração, no entanto após ter sido ultrapassada facilitou a forma como iríamos desenvolver o prototipo da interface gráfica e levou-nos a uma melhor compreensão sobre a implementação das diversas funcionalidades. Por fim, chegou a altura de desenhar o primeiro prototipo. Focámo-nos essencialmente em conseguir arranjar maneiras de implementar todas as nossas funcionalidades que tínhamos previstas seguindo os nossos Use Cases. De um modo geral, esta modelação ajudou-nos a prever futuros problemas e contribuiu para conseguir formular aquilo que queremos para o futuro da aplicação.

No que diz respeito à segunda parte do trabalho, podemos concluir que foi bastante mais trabalhosa, uma vez que tivemos que definir todos os diagramas de sequência de sistema e os de subsistema. Posteriormente, tivemos ainda que modelar os de implementação, que se tornaram bastante complexos e de difícil idealização. Partindo do modelo de domínio e dos modelos de sistema e implementação, pudemos definir o diagrama de classes, o qual, à medida que íamos desenhando os de implementação, fomos atualizando o diagrama de classes com os métodos e atributos respetivos a cada classe. Aquando da inicialização da implementação, seguindo os diagramas, foi um processo muito menos dispendioso, pois possuíamos já uma base fundamentada e estruturada para cada um dos requisitos que queríamos implementar, sendo os Use Cases a base com que trabalhamos. Porém, após termos implementado a UI e apesar desta cumprir os requisitos, reparamos que algumas interação não correspondiam ao descrito nos Use Cases anteriormente definidos, pelo que tivemos que alterar a interação do sistema com o utilizador, de modo a ser realizada como tinha sido idealizada e modelada. Em suma, a modelação do programa realizada por etapas nas fases precedentes à implementação, apesar de ter sido bastante dispendiosa e elaborada, permitiu-nos poupar tempo na construção física do sistema em si, permitindo-nos que este apresentasse menos erros (sendo estes apenas relativos à codificação das interações da UI).

Em suma, a aplicação por nós construída foi, na opinião do grupo, muito bem conseguida, pois, além de ser apelativa e de fácil utilização, respeita e cumpre todos os requisitos propostos no enunciado do trabalho, bem no início do projeto. Os modelos requeridos foram todos apresentados da forma mais correta que conseguimos, e foram utilizados como pilares para a implementação da aplicação em si. Estamos, portanto, perante um sistema totalmente utilizável no mundo real, e que poderia facilitar a configuração personalizada de um carro novo.