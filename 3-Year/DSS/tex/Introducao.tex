\section{Introdução}
A primeira fase do projeto da UC Desenvolvimento de Sistemas de Software consistiu no desenvolvimento dos modelos de domínio, use cases e da interface com o utilizador para a aplicação ConfiguraFácil. 

Na segunda fase, desenvolvemos os diagramas de sequência, de pacotes, de implementação e de classes. Além disso, procedemos também a implementação da aplicação em si, utilizando os modelos desenhados previamente. Para tal, utilizamos DAOs para podermos fazer a conexão entre a base de dados e a aplicação em JAVA. Alterámos também alguns modelos que apresentamos na fase anterior, corrigindo para uma versão que mais se foca nos objetivos da aplicação que pretendíamos modelar.

A aplicação \textit{ConfiguraFácil} consiste numa ferramenta existente nos stands de automóveis, que permite junto dos clientes criar uma configuração para uma encomenda de um carro novo. A aplicação guia o cliente em cada fase da configuração, permitindo-lhe escolher componentes individuais ou pacotes pré-definidos. 

Da perspetiva do grupo, a aplicação apresenta os seguintes requisitos:
\begin{itemize}
	\item O cliente pode escolher a pintura, jantes e pneus, motorização e detalhes interiores e exteriores;
	\item O cliente pode também escolher um pacote pré-definido que consiste num agregado de componentes individuais;
	\item Cada componente deve ter uma designação, preço, lista de componentes incompatíveis e lista de componentes complementares;
	\item Sempre que um componente é adicionado à configuração, a aplicação deve verificar se existe alguma incompatibilidade com algum componente previamente selecionado. Se tal existir, o cliente pode optar por desistir da seleção feita ou remover o produto incompatível. Além disso, deve também verificar se é necessário instalar algum componente complementar. Caso aconteça, o cliente pode manter a opção e instalar os componentes necessários ou então desistir da seleção;
	\item Quando um pacote pré-definido é selecionado, devem ser feitas as verificações de dependência/incompatibilidade para cada um dos componentes do pacote;
	\item Deve haver descontos associados aos pacotes, ou seja, os pacotes devem ser mais baratos que a soma os preços individuais de cada um dos seus componentes;
	\item Se o cliente selecionar individualmente todos os componentes que compõem um pacote, a aplicação deve reconhecer tal pacote e aplicar o respetivo desconto;
	\item Após as escolhas básicas, como a pintura e a motorização, o cliente deve poder indicar um orçamento para a encomenda e o sistema deve propor a melhor configuração possível dentro do orçamento apresentado. Ou seja, a aplicação deve conseguir gerar uma configuração ótima dado um orçamento;
	\item Cada componente deve ter um stock associado;
	\item Sempre que chega um novo stock de componentes, o sistema deve conseguir determinar quais são os carros que podem ser produzidos;
	\item Os carros são produzidos por ordem de chegada à fila de configurações efetuadas pelos clientes.
\end{itemize}
Desta forma, procuramos cobrir todos estes requisitos, de forma a que a aplicação seja o mais completa e coesa possível. Além do referido, incluímos também as funcionalidades de registar, identificar, consultar e alterar clientes, login de funcionários (sendo que o administrador pode registar, remover ou alterar os dados destes) e as funcionalidades relativas à gestão de fábrica de registar entrada de stock no sistema e produzir encomendas.